\documentclass[jou]{apa}
\usepackage{graphicx}
\usepackage{pdfpages}
\usepackage{fullpage}
\usepackage[english]{babel}
\usepackage{amsmath}
\usepackage{apacite}
\usepackage{url}
%Kopf- und Fußzeile
\usepackage{fancyhdr}
%\pagestyle{fancy}
%\fancyhf{}
%Kopfzeile links bzw. innen
\shorttitle{Weight of social information}
\renewcommand{\headrulewidth}{0pt}
\title{Integrating social and individual information.}
\author{Ulf Toelch}
\abstract{Social information exerts a strong influence on decision-making. For this, individual experiences are integrated with information obtained from the actions or products of others. When faced with such an integration task individuals in previous exhibited inter individual differences in their propensity to use social information often neglecting social information leading to disadvantageous and suboptimal decisions. Here, we investigate this integration process by building on and extending models on Bayes optimal integration of sensory information. We modeled individual strategies in a novel perceptional task that required integration of individual and social information. We show that individuals decide near Bayes optimal when integrating two social information sources but systematically deviate from Bayes optimal choice when integrating individual and social information.}
\affiliation{Mind and Brain \\ Humboldt University}
%\renewcommand{\figurename}{ESM Figure}
%\renewcommand{\tablename}{ESM Table}
\usepackage{hhline}
\begin{document}
\maketitle
\section{Introduction}
Every day we perceive a constant stream of social information that fundamentally influences our decision-making. Hence it is not surprising that major parts of our brain are involved in decision-making in social contexts \cite{rilling_neuroscience_2011,frith_mechanisms_2012}. Social information is frequently obtained via observation and copying actions of others will result in the acquisition of advantageous behavioural traits \cite{rendell_why_2010}. When observed actions and their outcomes are closely conjoined stimulus response contingencies are associatively learned with prediction errors similar to those found for associative learning from non social stimuli \cite{behrens_associative_2008, behrens_computation_2009,heyes_whats_2011}. \\
Frequently, observed behaviours are, however, not directly connected to an outcome. In mate choice copying for example it is not feasible to link long term relationship success or offspring survival to particular traits that are copied \cite{little_social_2011}. And also in novel situations we often follow social information without seeing apparent benefits closely adhering to the proverb ``When in Rome...'' \cite{reader_social_2008}. This bias for social information  \cite{mesoudi_bias_2006} is modulated by context and individual predispositions \cite{efferson_conformists_2008, toelch_decreased_2009}.\\
Along these lines Bahrami and colleagues \cite{bahrami_optimally_2010} recently performed an experiment to test whether joint decision-making, combining two individual information sources, would increase accuracy in a perceptional task. For this two participants first had to decide whether a conspicuous gabor patch appeared in the first or second set of six gabor patches. After an individual decision was made both participants were allowed to make a joint decision. This joint decision increased the accuracy when individuals had similar individual accuracies. The results were independent of whether participant had feedback on the outcome but depended solely on the exchange of confidence statements. Several questions, however, remain still open. In the aforementioned experiments participants needed to make a joint decision when information conflicted in a 2AFC task. This does however not encompass a measurement of how social information is integrated on an individual level. That is, how do humans process social information (compared to individual information) and how are decisions influenced not only in joint but also individual decisions. One example comes from soccer where two referees (main and assistant) have to estimate whether a player at the time of a pass was offsides or whether a foul was conducted. The main referee has his own due to his position noisy information and the assistant referee may indicate that the player's position was offside or that a foul was conducted. The main referee may then follow the decision by the assistant or decide to overrule this decision. For this he has to integrate his own information and the social information he received from his assistant and come to a final decision. Moreover, participants in the aforementioned study communicated directly with each other leaving room for interpretation whether the observed effects were due to normative or informational influences. To fully understand how human individuals integrate social information with their own information (which forms the potential basis for joint decision making) it is necessary to investigate how individuals gradually change their opinion or are swayed by the choice of others in a purely informational context. In our experiment players had to solve a perceptual task where they had to guess the location of a stimulus that briefly flashed on the screen. In the first phase players could assess their own accuracy and the accuracy of two other players in this task by receiving feedback on the correct location after choosing. In the second phase participants received information generated in the first phase and made a second guess on the position of the stimulus without receiving feedback on the actual position of the stimulus. 

\section{Method}
\subsection{Participants}
23 Participants with an average age of 28.1 years (\textit{SD}=7.4, 12 female) participated that were recruited via posters and paid on average 10 Euro based on their performance in the game, that is how well they guessed the position of the stimulus during the experimental phases. Before beginning, written instructions were given and participants were allowed to ask questions aloud and written informed consent was obtained from all participants. After the experiment, participants were paid in private and were debriefed. The procedures and questionnaires were approved by the Medical Ethics Review Committee of the Charite University Hospital  (protocol number EA1/212/11) and comply with the ethical guidelines of the APA.

\subsection{Experimental Game}
The game consisted of an initial training phase (40 trials) and two experimental phases (80 and 140 trials). The training phase and first phase involved guessing the location of a stimulus that briefly flashed on the screen whereas in the second phase players had to make a second guess based on their guesses from the first phase. 

\subsubsection{Training phase and first experimental phase}
Each trial was commenced by pressing a key and after an initial 1 s to 2 s break a stimulus (filled circle) was briefly flashed for a short time interval (50 ms) on the outline of a pre-drawn circle on a computer screen. Immediately after the stimulus presentation distractor stimuli are displayed on the outline of the empty circle for 3 s. During this interval the mouse pointer slowly moves from the center to the rim. Players are instructed to keep the pointer in the center of the screen or otherwise they would loose points. This manipulation prevents players from executing a saccade to the target location and simply keep their gaze fixed on the target. Instead they have to fix their gaze at the center again. Participants then had to make a guess where on the outer empty circle the stimulus appeared. The location of the first guess is displayed as a small red circle. In the first experimental phase players will see, additional to their own choice, a green and a yellow circle representing the choices of two other players that were recorded in a pilot study and the actual position of the stimulus. The other players were selected based on their accuracy with one player being more accurate than the other. Players were not take from the extremes of the available players from the pilot study but had distinct accuracies with a mean deviation of 12.7 degrees for one player and 5.7 degrees from the actual stimulus position for the other social player (Figure 1a). 
\subsubsection{Second Experimental Phase}
Again each trial was started by pressing a key. In this phase players viewed the choice from randomized trials from the first experimental phase. They always received two pieces of information to inform a second guess on where the stimulus was located. There were four information pairs; their own guess paired with a guess of either of the social players. Additionally players received the information from the two social players without their own information. They also received pairs that consisted only of their own guesses. This was possible since we presented players with a set of 60 stimuli in the first experimental phase. Twenty stimuli were repeated and randomly interspersed so that for twenty stimuli we had two individual guesses of the players available.
\begin{figure}
\includegraphics[width=0.5\textwidth]{Figure_Methods}
\caption{In phase 1 (80 trials) players assessed their own accuracy and the accuracy of the other players in a perceptual task where they had to guess the position of a briefly flashed stimulus. After observing the stimulus for 50 ms players several distractor stimuli appeared in quick succession in random places along a circle. During the whole time players had to center their mouse pointer that was slowly moved by the computer in one direction. After this players had to guess the position of the stimulus and then saw the decisions of two other players as well as the actual position of the stimulus. In phase 2 players received information from phase 1. They always saw a combination of two guesses consisting of either individual information (red) or social information (green, yellow). Based on this information players made a second guess on the position of the stimulus. In the second phase players received no feedback.}
\end{figure}
\subsection{Model fitting}

We modeled player decisions in the second phase via maximum likelihood methods to individual player data for decisions that involved individual information and one of the social information types. Generally, players had two different types of information available. (I) the distance between the two displayed guesses and (II) their estimate of their own accuracy and the accuracy of the social players. Based on this information they made a second guess that deviated by a certain degree from their initial individual choice. The models differ with regard to whether and how these two information types are taken into account when determining how far player decisions deviated from the initial choice ($D_{ind-fin}$). We restricted our analysis to cases where the individual deviation and the final deviation were below 30 degrees.
\subsubsection{Constant deviation($CD$)} In this model players largely ignore the available information and the deviation from the initial guess shows a constant deviation that is independent of the distance between the two guesses. In linear regression terminology this would be an intercept only model (\ref{eq:M1}).
\begin{equation}
D_{ind-fin}=k
\label{eq:M1}
\end{equation}
\subsubsection{Mean deviation ($MD$)} Players engaging in this strategy will take the distance between the two guesses into account ($D_{ind-soc}$) but not the different accuracies of each player. This will lead to higher deviations with higher distance between the two guesses that are similar for decisions that involve a social player independent of the accuracy (\ref{eq:M2}).
\begin{equation}
D_{ind-fin}=k*D_{ind-soc}
\label{eq:M2}
\end{equation}
\subsubsection{Bayes Optimal ($BO$)}
This model describes a Bayes optimal strategy as has been described for the integration of information in multi-modal perception tasks  \cite{alais_ventriloquist_2004}. The deviation of the final guess from the initial guess is weighted ($\omega$) by the accuracy of the available information ($acc_{ind}, acc_{soc}$) measured as the inverse variance of individual guesses around actual position during phase 1. One possibility is that players accurately estimated the accuracy of the social information and decided Bayes optimal (Equation \ref{eq:M3} with \ref{eq:M4a}). Then the weight changes based on the accuracy of the social player. More weight is given to individual information when the accuracy of the social players is low. However, it is possible that players misjudged the accuracy of the two social players due to a high cognitive load or limited time in phase 1 and used the wrong estimates in a Bayes Optimal manner (Equation \ref{eq:M3} with \ref{eq:M4a} and \ref{eq:M4b}). To allow for this possibility we calculated a correction factor ($c_{soc}$) for the player estimate of the social players' accuracy from the decision players made in trials where they saw two guesses based on social information. We derived this factor from the regression coefficient of a linear regression (intercept set to 0). This regression correlated the distance between the two social player guesses with the distance between the guess of the more accurate player to the final guess of the player (see Appendix 1 upper left panel). We then adjusted the accuracy of the player with low accuracy by the formula given in equation \ref{eq:M4b}.
\begin{equation}
D_{ind-fin}=\omega * D_{ind-soc}
\label{eq:M3}
\end{equation} 
\begin{subequations}
\begin{align}
\omega=\frac{acc_{ind}}{acc_{ind}+acc_{soc}}\label{eq:M4a}\\
acc_{soc_{low}}=\frac{acc_{soc_{high}}*c_{soc}}{1-c_{soc}}\label{eq:M4b}
\end{align}
\end{subequations}
\subsubsection{Bayes Optimal with individual discounting ($BO_{IND}$)}
This model is similar to the $BO$ model but allows for the possibility that players valued their own information systematically better or worse than the information coming from the social players. We added one more free parameter (k) that scaled the individual accuracy (Equation \ref{eq:M3} with \ref{eq:M5}). The $BO$ model is a special case of the $BO_{IND}$ model with k fixed at 1.
\begin{equation}
\omega=\frac{k*acc_{ind}}{k*acc_{ind}+acc_{soc}}\label{eq:M5}
\end{equation}
\subsubsection{Two slopes ($TS$)}
As a comparison we fitted a linear regression (origin fixed at zero) allowing for different slopes depending on which social information (player one or two) was presented together with the individual information. This model explained more variance than the other models (except in some cases where $WTA$ was the best fitting model) but did so at the expense of fitting two free parameters. We will discuss this in the model selection section below.
\subsection{Model selection}
We fitted all models to individual data. From the maximum log likelihood ($\ell$) of each model we calculated the Bayesian Information Criterion (Equation \ref{eq:M6} with  $m$ giving the number of free parameters and $n$ the number of data points). For the $BO$ and $BO_{IND}$ models we fitted several sub models (see above). Before comparing models we selected the model with the lowest BIC from these sub models. This preselection of models was necessary to keep a balance between the number of models that are based on a certain model type when calculating the BIC weights in the next step. For the remaining five models we calculated the BIC weights to have a comparable evidence for each player for each model\cite{bolker_ecological_2008}.
\begin{equation}
BIC=-2*\ell+m*ln(n)
\label{eq:M6}
\end{equation} 

\subsection{Questionnaire and psychological scales}
After the experiment participants filled out a questionnaire with items regarding general demographic parameters as well as specific questions regarding the experiment (see Appendix 2, not finished yet) Additionally, we will included scales on individualism collectivism \cite{oyserman_rethinking_2002} and the perspective taking sub-scale of the empathy scale \cite{davis_multidimensional_1980}.

\section{Results}

All players completed the game but players 1 and 20 had excessively high deviations ($M_{id=1}=107 ^\circ, M_{id=20}=120^\circ$) for their individual guesses so that we excluded them from the subsequent analyses. The average deviation of the remaining 21 players for their first guess was $M=14.0^\circ$ ($SD=6.9$). This deviation was significantly reduced to $M=10.0^\circ$ ($SD=5.5$) for their final guess (Wilcoxon paired signed rank test: $V=49, p=0.014$). 
We fitted the models to each individual as outlined in the Methods section and obtained for each player evidence which model describes the player's strategy best given our suite of models. 
\begin{table}[ht]
\begin{center}
\caption{BIC weights for the Constant Deviation ($CD$), Bayes optimal ($BO$), Bayes optimal with correction for individual information weight ($BO_{IND}$), Mean Deviation ($MD$), and two separate slopes ($TS$) (see Methods for details). Maximum weights are displayed bold. }
\begin{tabular}{lccccc}
  \hline
 ID& WTA & BO & BO\_IND & MD & TS \\ 
  \hline
2 & 0.00 & 0.00 &\textbf{ 0.83} & 0.06 & 0.10 \\ 
  3 & 0.00 & 0.00 & 0.29 & \textbf{ 0.63} & 0.08 \\ 
  4 & 0.00 & 0.00 & \textbf{ 0.89} & 0.00 & 0.11 \\ 
  5 & 0.02 & 0.44 & \textbf{ 0.45} & 0.00 & 0.09 \\ 
  6 & 0.00 & 0.00 & \textbf{ 0.89} & 0.00 & 0.11 \\ 
  7 & 0.00 & 0.00 & \textbf{ 0.88} & 0.00 & 0.12 \\ 
  8 & 0.00 & \textbf{ 0.49} & \textbf{ 0.49} & 0.00 & 0.01 \\ 
  9 & \textbf{ 0.78} & 0.00 & 0.14 & 0.03 & 0.05 \\ 
  10 & 0.00 & \textbf{ 0.47} & \textbf{ 0.47} & 0.00 & 0.06 \\ 
  11 & 0.23 & 0.00 & 0.12 & \textbf{ 0.45} & 0.20 \\ 
  12 & \textbf{ 0.99} & 0.00 & 0.00 & 0.00 & 0.00 \\ 
  13 & 0.00 & 0.00 & 0.13 & \textbf{ 0.76} & 0.11 \\ 
  14 & 0.00 & 0.00 & \textbf{ 0.75} & 0.00 & 0.24 \\ 
  15 & 0.00 & 0.00 & \textbf{ 0.78} & 0.10 & 0.12 \\ 
  16 & \textbf{1.00} & 0.00 & 0.00 & 0.00 & 0.00 \\ 
  17 & 0.00 & \textbf{ 0.46} & \textbf{ 0.46} & 0.01 & 0.07 \\ 
  18 & 0.00 & 0.00 & 0.04 & \textbf{ 0.83} & 0.12 \\ 
  19 & 0.00 & 0.32 & \textbf{ 0.58} & 0.00 & 0.11 \\ 
  21 & 0.00 & 0.00 & \textbf{ 0.87} & 0.00 & 0.13 \\ 
  22 & 0.16 & 0.00 & \textbf{ 0.65} & 0.02 & 0.17 \\ 
  23 & 0.01 & 0.05 & 0.05 & \textbf{ 0.79} & 0.10 \\
   \hline
\end{tabular}
\end{center}
\end{table}
\section{Caption Appendix 1}
\textit{\textbf{Upper left panel:} Distance between the two social guesses (abszissa) plotted against the distance between the final guess and the guess of the more accurate player (ordinate). The fitted line shows a linear regression through the origin.  With an increase in slope players used the information of the less accurate player more. (The line for Bayes optimal choice will be added in future version).\\ \textbf{Upper right panel:} Distance between individual guess and guess of one social player (abszissa) and distance between individual guess and final guess (ordinate). The social player with higher accuracy is shown in the right panel. The fitted lines correspond to the best fitting model (see also Table).\\ \textbf{Table (lower left)}: Raw BIC values for the different models described in the Methods. The star denotes the lowest BIC i.e. the best fitting model.\\
Other right panels still experimental... }

\bibliography{cite}




\end{document}